\chapter{Einleitung}
Das Rubik’s Cube Projekt wurde im Rahmen des Moduls Spieleprogrammierung - Vertiefung entwickelt. Ziel des Projekts war es, einen interaktiven Rubik’s Cube zu simulieren, der mithilfe von Maus- und Tastatureingaben gesteuert werden kann. Die technische Umsetzung erfolgte unter Verwendung von C++ und OpenGL, wobei mathematische Konzepte wie Quaternionen und Transformationen zur Anwendung kamen.

In dieser Hausarbeit werden die Funktionsweisen der wichtigsten Klassen und die grundlegenden Lösungsideen dokumentiert. Der Fokus liegt auf den entwickelten Ansätzen sowie einer oberflächlichen Beschreibung der Implementierung.

\section{Steuerung}
Die Interaktion mit dem Rubik’s Cube erfolgt über folgende Bedienelemente:

\begin{itemize}
    \item Rechte Maustaste: Halten und Bewegen der Maus rotiert den gesamten Würfel im 3D-Raum
    \item Linke Maustaste: Halten und Bewegen der Maus ermöglicht das Drehen einzelner Slices (Zeilen und Spalten).
    \item Leertaste: Setzt den Würfel in den Ausgangszustand zurück.
    \item Scrollrad: Ermöglicht das Zoomen mit der Kamera.
\end{itemize}
