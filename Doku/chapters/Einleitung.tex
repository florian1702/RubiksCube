\chapter{Einleitung}
Das Rubik’s Cube Projekt wurde im Rahmen des Moduls Spieleprogrammierung - Vertiefung entwickelt. Ziel des Projekts war es, einen interaktiven Rubik’s Cube zu simulieren, der mithilfe von Maus- und Tastatureingaben gesteuert werden kann. Die technische Umsetzung erfolgte unter Verwendung von C++ und OpenGL, wobei mathematische Konzepte wie Quaternionen und Transformationen zur Anwendung kamen.

In dieser Hausarbeit werden die Funktionsweisen der wichtigsten Klassen und die grundlegenden Lösungsideen dokumentiert. Der Fokus liegt auf den entwickelten Ansätzen sowie einer oberflächlichen Beschreibung der Implementierung.

\section{Steuerung}
Der Rubik’s Cube wird vollständig durch Maus- und Tastatureingaben gesteuert:

\begin{itemize}
    \item Rechte Maustaste gedrückt: Durch Ziehen wird der gesamte Cube rotiert.
    \item Linke Maustaste gedrückt: Ermöglicht das Drehen einzelner Slices (Zeilen und Spalten).
    \item Leertaste: Setzt den Würfel in den Ausgangszustand zurück.
    \item Scrollrad: Kann benutzt werden, um mit der Kamera herein- oder herauszuzoomen.
\end{itemize}
